\documentclass{slides}
\usepackage{url}
\title{PSNA Chapter 26: Perception and Visibility}
\author{Steven L. Jenkins}
\newcommand{\bi}{\begin{itemize}}
\newcommand{\ei}{\end{itemize}}
\newcommand{\bd}{\begin{description}}
\newcommand{\ed}{\end{description}}
\newcommand{\be}{\begin{enumerate}}
\newcommand{\ee}{\end{enumerate}}

\begin{document}

\slide{Perception and Visibility}

\bd
\item[Peception] is how people see you; it is a measure of quality
\item[Visibility] is how much people see of you; it is a measure
  of quantity
\ed

\emph{A customer's peception of you is their reality of you.}

\slide{The Basics}

\bi
\item You get exactly one chance to make a good first impression
\item Smile: do you want to be known as \emph{Grumpy}, or as someone
  who comes up with positive solutions to problems?
\item \emph{A nonexasperated, unruffled exterior is the greatest asset an
  SA can have}
\ei

\slide{First Day}

\bi
\item Have the desktop ready
\item Have the account set up
\item Send your friendliest person to greet/orient
\item If you don't introduce the team, Someone Else Will
\ei

\slide{Attitude}

\bi
\item \emph{The number one attitude problem among SAs is a blatant disrespect
for the people whom they are hired to serve.}
\item Replace \emph{users} with \emph{customers}
\item Remember the reality that people will bring you problems; if that's
  not the job you want to do, consider a different job.
\item Take initiative to get to know your customers as capable in their
  areas.  This helps build mutual respect.
\ei

\slide{Priorities vs Expectations}

\bi
\item Emergencies now
\item Small jobs quickly
\item Large jobs take longer time
\ei

But the definitions of \emph{large} vs \emph{small} can be 
different.  You may find it helpful to work out a concrete SLA.

\slide{SLA}

A Service Level Agreement is a document that specifies the 
expected length of time to complete common tasks.

Example:

\bi
\item Create user account
\item Install OS
\item Change password
\item Install software
\ei

\slide{SLA: cont'}

If you find your customers are unhappy with your group's performance, it
may be helpful to work out a concrete SLA with them, as a way of
better understanding their expecations.

\slide{Be the System Advocate}

Rather than a \emph{system clerk}, be a \emph{system advocate}.  Be
pro-active and anticipate the needs of your customers.  Become a
\emph{can-do} person.

\slide{Advocate: Installing Software}

\bd
\item[Clerk:] receives software, works through numerous configuration and 
  installation issues, feels like a hero, receives negative feedback 
  because of delays.
\item[Advocate:] knows customers needs certain software, works with 
  customer to help choose best package, helps arrange appropriate licensing
  terms, ensures adequate capacity is available, establishes agreed-upon
  timetable, customer better understands process (i.e., it is more
  transparent), customer better satisfied (or at least understands what
  to do if she wants better service).
\ed

\slide{Advocate: Solving Performance Problem}

\bd
\item[Clerk:] reboots, upgrades all aspects of a system, customer not
  happy, management not happy because of high costs and low performance.
\item[Advocate:] monitors systems and sees problem before the customer,
  presents upgrade options and/or re-architecture suggestions to improve
  performance, customer more satisfied, management much more satisfied
\ed

\slide{Advocate: Automation}

\bd
\item[Clerk:] does everything by hand, works long hours, bored with job
\item[Advocate:] automates everything possible, understands the appropriate
  programming languages, but always willing to \emph{buy} instead of
  building her own, has time to grow and stay motivated.
\ed

Automation of command-line operations is easy: use {\tt script} to get 
started.  Automation of GUI operations is much, much harder.  This is why
most SAs prefer command-lines.

\slide{Icing}

One of the key problems of SAs is the Visibility paradox:
address that via marketing

\bi
\item System status web page
\item Management meetings
\item Physical visibility
\item Town meetings, but plan them well
\item Newsletters (but be careful)
\item Mail to all customers (in newspaper style)
\item Special events/prizes (e.g., lunch, after-hours events)
\ei

\end{document}
