\documentclass{slides}
\usepackage{url}
\title{The Practice of System and Network Administration: Chapter 8}
\author{Steven L. Jenkins}
\newcommand{\bi}{\begin{itemize}}
\newcommand{\ei}{\end{itemize}}
\newcommand{\bd}{\begin{description}}
\newcommand{\ed}{\end{description}}
\newcommand{\be}{\begin{enumerate}}
\newcommand{\ee}{\end{enumerate}}
\newcommand{\assignment}[1]{{\bf Idea: } #1 }

\begin{document}

%\maketitle

\slide{PSNA Chapter 8: Disaster Recovery and Data Integrity}


Examine what may occur, prioritize, and plan.

Part of an overall business plan, not just technology (SG anecdote)

\bd
\item[BCP] is Business Continuity Planning
\item[DRIP] is Disaster Recovery Implementation Planning
\ed

\slide{What is a disaster?}

Anything that disrupts your business.  See \url{http://www.23.com/backhoe}
for an example.

\slide{Risk Analysis}

Budget is $(cost_{disaster} - cost_{mitigation})\times risk$

\bi
\item $Cost_{flood} = 10000000$
\item $Pr(flood) = 10^{-6}$
\item implies the budget should be \$10, assuming that the cost after
mitigation would be zero.
\ei

\slide{Legal Obligations}

\bi
\item to owners (e.g., stockholders)
\item to organizations (e.g., SEC, governmental bodies)
\item to clients (e.g., contractual agreements)
\ei

\slide{Damage Limitations}

Reduce the impact of a disaster.

Sidebar: fix once vs recurring cost

\slide{Preparation}

The day of the disaster should not find you unprepared.

$C^3$ is key

\slide{Data Integrity}

\bi
\item Change management and access control
\item Read-only implies checksums
\item unexpectedly large changes
\item unexpected changes (e.g., lockdown windows)
\item {\tt perl -c}
\ei

\slide{Redundancy}

\bi
\item hot-hot vs hot-cold
\item Security disaster
\item media relations
\ei

\slide{One Way To Pay For DR}

Iff

\bi
\item MAN is cheap (or already paid for)
\item HA is already paid for
\item your company has multiple data centers
\ei

then

\bi
\item split your HA pairs
\ei

\end{document}
