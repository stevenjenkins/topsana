\documentclass{exam}
\usepackage[dvips]{graphicx}
\usepackage{comment}
%\linespread{1.3} % for one and a half
%\linespread{1.6} % for double-spacing
\newcommand{\myq}{\question[1]}
\addpoints
\begin{document}
%\maketitle
{\large CS 5360 Final Exam 2003: Due April 30, 8am, hardcopy to my office}\\
%\vspace{0.2in}
\hbox to \textwidth{\large Name:\enspace\hrulefill}
\vspace{0.1in}

This exam has \numquestions{} questions for a total of \numpoints{} points.

Note that you may choose to do this exam as a cumulative exam, covering
all chapters from PSNA.  If so, this exam will replace your mid-term, as well
as count as the final.

\begin{questions}

\question[100]

\section*{Introduction}

Congratulations!  You have just been hired by a company which is looking
to expand.  You have been asked to provide Systems Administration
consulting directly to the CEO, who wants you to look over the company's
business plan and make suggestions about the company's IT infrastructure
needs.  Your presentation should both educate the CEO about the basic
options, as well as make recommendations.  In addition, you are asked 
to prepare training materials suitable for the setup of the new IT staff.

You are specifically asked to cover the business plan.  It may be that some
of the materials presented in the business case below are too ambiguous, or 
are simply impractical.  Your job will be to clearly document some reasonable
assumptions to make, and build your presentation on that.  If the business
case is impractical, you will need to clearly explain the IT portion of
the plan that is not feasible and make a counter-proposal that is
more appropriate.  {\bf Hint:} the IT budget (personnel, capital for 
equipment, and any monthly charges for services) must be less than the
expected revenue of the company.  

Note that the company has contracted for multiple presentations -- sharing
of presentation materials across the consultants is forbidden by contract.
\emph{Translation: all work should be done by individuals with no discussion
of the exam, per the Code of Conduct.}

To determine which of the cases you should use, simply take the remainder,
modulo 2, of your student id and match that with the appropriate case (i.e.,
odd numbered students should do number 1; even numbered students should
do number 2).  The third case below is for those that are interested 
in looking at a more fundamental IT service company.  {\bf You must
clearly identify which case you are using.}

For your reference, I have included the chapter titles of the last
16 chapters of PSNA:

\begin{comment}
\begin{enumerate}
\item Desktops
\item Servers
\item Services
\item Debugging
\item Fixing Things Once
\item Namespaces
\item Security Policy
\item Disaster Recover and Data Integrity
\item Ethics
\item Change Management and Revision Control
\item Server upgrades
\item Maintenance Windows
\item Service Conversions
\item Centralization and Decentralization
\end{enumerate}
\end{comment}

\begin{itemize}
\item Helpdesks
\item Customer Care
\item Data Centers
\item Networks
\item Email Service
\item Print Service
\item Backup and Restore
\item Software Depot Service
\item Service Monitoring
\item Organizational Structures
\item Perception and Visibility
\item Being Happy
\item A Guide for Technical Managers
\item A Guide for Nontechnical Managers
\item Hiring System Administrators
\item Firing  System Administrators
\end{itemize}

Note that you may need to pull information from a few sections in the
first part of PSNA for this -- specifically, you'll need to provide
specific proposals and cost tradeoffs for the desktops, servers, and 
services to the CEO, but
you do not need to cover the IT material (unless you are doing the 
cumulative exam).  

\section*{Business Models}
\begin{comment}
\begin{enumerate}
\item Large company with 5000 employees on a 3-building campus handling 
  the back-end of an
  e-commerce site.  There is an in-house development team of 10 developers,
  and the full-time, permanant IT staff for internal needs should have no 
  more than an additional 20 people.  You do not need to consider
  the Help Desk for external users.  Note that each person in the
  company has a desktop PC.  There are 4 servers for development, and
  4 servers for production use by the e-commerce site.  
  Other servers (and services) need to  be specified by you.
\item Small engineering company with 200 employees, 150 of which are
  developers (on both Windows and Unix platforms), 30 of which are 
  sales \& marketing, with the final
  20 being management.  The company is heavily influenced by Extreme
  Programming and has broken up the developers into 20 small 
  departments with very differing needs.  You will need to make a 
  recommendation for the
  size of the IT team in addition to the other requirements.  Each
  engineer has 2 machines on her desk: one running Windows and one
  running either Solaris, FreeBSD, or Linux (you can assume roughly
  $1/3$ are running each platform).  
\item Small university of 1000 students, 50 faculty, and 
  20 administrative staff (including 5 IT).  Each faculty and staff
  member has a desktop PC, but no students have individual machines: they
  share the use of 4 labs with 50 machines each that are open 24 hours.
  
\end{enumerate}
\end{comment}

\begin{enumerate}
\item Kao's Kreations -- a sandwich shop with 100 stores (and 1200
  employees: home office has 200 and each store has 10) scattered throughout
  East Tennessee, Southwest Virginia, and Eastern North Carolina.  The main
  office is location in Johnson City.  Each store must connect back to
  the main office for each transaction (e.g., all credit card clearing
  is done at the main office, as is all inventory management), but
  the bandwidth needs are quite low.  This 
  company is marginally profitable, but they want to understand how much
  leverage they can gain by using IT more effectively.  The default IT
  size is 4 people.  You may assume the cost per month of operating each
  store (including consumables, rent, staffing, and all incidentals) is 
  10K USD, and the income is 15K USD per month per store.  The home office 
  cost per month is 200K.  

  After you have completed the basic assignment for the CEO and for the
  IT staff, the CEO would like for you to present a few slides (no more
  than 4) on a new service:  management is considering adding Streaming
  Video into each store as a way of attracting new customers and encouraging
  customers to spend more time in the store.  You can assume the content
  costs are free, but you need to consider the networking costs.  You 
  do not need to worry about presenting any materials on Streaming Video
  to IT.  You also don't need to worry that the requirements from the CEO
  are very vague -- the CEO simply wants to understand an entry point
  for Streaming Video, and understand some basic concepts.  You may assume
  that a single low-end server is sufficient for distribution and that
  300K per stream is required.  You don't know if the CEO wants to stream
  one stream or multiple streams per office, so you should prepare a few
  datapoints (i.e., cost points) for the CEO to look at.
  
\item W33l Hack U2 is a small security company of 100 employees, each
  employee having 3 platforms as a desktop.  Approximately 20 staff
  are in Tiger Teams, performing penetration tests and analyses for
  external customers; 30 staff are titled \emph{developers} and
  create and support a real-time authorization
  and authentication system that is sold to several very large customers
  (which they access via the Internet),
  25 developers working on new security products, 5 IT support staff, 
  and 20 sales and management.  

\item {\bf Challenge Question (optional):} Small, regional ISP in
  Northeast Tennessee (total population: 500,000) with 
  approximately 1000 customers each currently paying \$20 per
  month for basic access.  The primary IT infrastructure envisioned is
  network connectivity (don't forget you have to pay for both ends of the
  connections, but you can assume there is some oversubscription of 
  service).  This company has been losing money regularly, and part of your
  job is to explain some of the economics of the situation to them and 
  propose additional revenue streams.  The default IT size is 2 people, and
  the management is 2 people.  The current owners want to have a revenue
  stream of 4 million US per year within 3 years, and cashflow should be
  positive by next year.

\end{enumerate}

\section*{Deliverables}

All materials should be suitable for presentation to an appropriate audience.

\begin{itemize}
\item An introduction to the material, describing which  of the last
  16 chapters in
  \emph{The Practice of System and Network Administration} are important
  for the CEO and which are important for the IT staff.
\item A slide presentation for the CEO, covering the appropriate items, 
  with appendices if necessary.  Your presentation for the CEO must include
  decisions for the CEO to choose between options.  You must also include
  costs.
\item Materials for the IT staff in whatever form you think most 
  appropriate for training and reference.
\end{itemize}

\section*{Hints}

For the purposes of this exam, you should use the costs below.  If you find
you need additional costs, please contact me so that I can distribute the
information to the rest of the class.

\begin{tabular}{|c|c|c|}
\hline
{\bf Personnel/Region} & {\bf Regional} & {\bf National}\\
\hline
SAGE 1 (novice) & 20K USD & 40K USD \\
SAGE 2 (junior) & 30K & 50K \\
SAGE 3 (intermediate) & 40K & 80K \\
SAGE 4 (advanced) & 60K & 120K\\
\hline
\end{tabular}

\begin{tabular}{|c|c|}
\hline
{\bf System} & {\bf Cost} \\
\hline
Low-end desktop & 1K \\
Mid-range desktop & 1.5K \\
High-end desktop & 2K\\
Low-end server(PC) & 2K \\
Mid-range server(PC) & 4K \\
High-end server(PC) & 15K \\
Low-end UNIX server & 2K \\
Mid-range UNIX server & 10K \\
High-end UNIX server & 100K \\
\hline
\end{tabular}

\vspace{0.5in}

\begin{tabular}{|c|c|}
\hline
{\bf Link speed} & {\bf Cost/month} \\
\hline
Low (modem) & \$20\\
Middle (DLS/Cable/T1) & \$30 \\
High (above T1) & \$400\\
\hline
\end{tabular}
\vspace{0.5in}

Note that the prices are \emph{not} accurate, but I have simplified this
to make it easier for your analysis.  Note in particular that it is 
inappropriate to directly
compare a \emph{high-end} server of PC hardware to a \emph{high-end} UNIX
server based on the words \emph{low-end}, \emph{mid-range}, and 
\emph{high-end}.  Part of your job will be explaining that to the CEO 
and helping to navigate the decision-making therein.  You may
assume that disk space is included in the price of the servers (that's
an unreasonable assumption, but it's simple).

While you are not allowed to discuss this with any other student, you are
free to ask me questions.  I would prefer for you to email me questions
directly, and I will post questions and answers to the discussion board
on Blackboard.

Also, you are free to use materials from elsewhere on this exam, but you
must provide citations before the material is included (e.g., at the start
of your inclusion).  Do not put all references merely at the end of your
document.


\end{questions}

\end{document}

