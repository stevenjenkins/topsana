\documentclass{slides}
\title{System Performance Tuning: Chapters 1 and 2}
\author{Steven L. Jenkins}
\usepackage{url}
\newcommand{\bi}{\begin{itemize}}
\newcommand{\ei}{\end{itemize}}
\newcommand{\bd}{\begin{description}}
\newcommand{\ed}{\end{description}}
\newcommand{\be}{\begin{enumerate}}
\newcommand{\ee}{\end{enumerate}}
\newcommand{\assignment}[1]{{\bf Idea: } #1 }


\begin{document}

\maketitle

\slide{Introduction}

There are two general areas we'll discuss: performance tuning and
capacity planning.  Both of these depend on computer architecture and
application architecture.

\bi
\item CPU speed
\item Memory capacity
\item Disk speeds
\item Network speeds
\ei


\slide{Principles}

\be
\item Understand your environment
\item TANSTAAFL (\emph{cheap, fast, reliable: pick two})
\item Throughput vs latency
\item Overutilization of a resource: \emph{The Wall}
\item Design tests carefully
\ee


\slide{Workflow Management}

\bi
\item Dynamic systems, not static
\item Heisenberg's Principle of Uncertainty
\ei

Solution: careful, methodical measurement.

\slide{Three Scenarios}

\bi
\item Startups
\item Governments
\item Utopia
\ei

\slide{Tools}

\bi
\item {\tt vmstat}
\item Process accounting ({\tt \verb+\etcinit.d\acct+} -- not on Linux)
\item {\tt sar} (not in default RedHat: look for sysstat package)
\ei

Note: the text uses Solaris and Linux, but Linux does not have nearly
as rich as a toolset.  Building performance tools on Linux is a very
good way to increase your understanding of systems.

\slide{Patterns}

\bi
\item Request-response
\item Inverse request-response
\item Data transfer
\item Message passing
\ei

Another interesting exercise is looking at packet size distributions.

\slide{Email}

Discuss:

\bi
\item What are the metrics interesting in email systems?
\item What are the expected patterns in email systems?
\item What is the expected packet size distribution?
\item What are the possible bottlenecks in the mail systems that 
  you have assembled?
\item How would you create tests to determine these?
\ei

\end{document}
