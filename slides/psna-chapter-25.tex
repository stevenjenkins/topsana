\documentclass{slides}
\usepackage{url}
\title{PSNA Chapter 25: Organizational Structures}
\author{Steven L. Jenkins}
\newcommand{\bi}{\begin{itemize}}
\newcommand{\ei}{\end{itemize}}
\newcommand{\bd}{\begin{description}}
\newcommand{\ed}{\end{description}}
\newcommand{\be}{\begin{enumerate}}
\newcommand{\ee}{\end{enumerate}}

\begin{document}

\slide{Organizational Structures}

How an SA group fits into the overall organization and how the SA group
is structured help determine the success of an SA organization.

\slide{Sizing}

\bi
\item Metrics are critical
\item There are no magic ratios
\ei

\slide{Basic Sizing Ratios}

\bi
\item Customer/desktop support
\item Customer/server support
\item Infrastructure support
\item SA to customer ratio
\item SA to server  ratio
\item SA to service ratio
\ei

\slide{Funding}

Understanding how the SA group is funded (and billed out) is your
responsibility.  This requires:

\bi
\item Understand the funding (i.e., business model)
\item Understand the billing (i.e., finance)
\item Communicate with management (i.e., CEO \& CFO)
\ei

Some SAs (and SA managers) neglect this \emph{non-technical} issue to their
own detriment.

\slide{Cost Centers}

SA groups are typically cost centers, not profit centers.  The funding
model is typically:

\bd
\item[Centralized] Economy of scale, but often less mobile and flexible
\item[Decentralized] High-touch, high flexibility; often higher cost
\ed

\slide{Management Chain}

\bi
\item CTO
\item COO or CFO
\item Individual BUs
\ei

\slide{Appropriate Skills}

\bi
\item Maintenance
\item Customer support
\item Deployment
\item Architecture
\item Senior generalists and integrators
\ei

\slide{One Model}

\bd
\item[Level 1] Help Desk (C)
\item[Level 2] BU-specific end-user support (D)
\item[Level 3] Infrastructure teams (C)
\ed

The Help Desk and infrastructure teams focus on the core technical 
problems across the organization.  The BU-specific teams give 
higher \emph{touch} to those groups that pay for it.

\slide{Outsourcing}

\bi
\item Low-needs implies commodity
\item High-needs implies high-touch
\ei

Keep in mind that there are three ways to do outsourcing:

\bi
\item Nothing
\item Everything
\item Hybrid
\ei

\slide{Outsourcing: Features}

\bi
\item Can provide commodity service for a company without SA skills
\item Multiple providers exist, so there is some flexibility
\item Allows a company to focus on core business
\ei

\slide{Outsourcing: Risks}

\bi
\item SA incentives not tied to company's success
\item Newer technology may not be covered
\item Security needs should be carefully specified
\ei

\slide{Outsourcing: Final Thought}

Question: once you've outsourced something, how do you go about
\emph{insourcing} (or changing vendors) if you change your mind later?

\slide{Consultants and Contractors}

\bd
\item[Consultants] Experts for temporary help
\item[Contractors] Backfill current SAs (or provide additional,
  temporary, non-specialist skills)
\ed

\slide{Sample Organizational Models}

The slides following detail some models.  As you think of these,
keep in mind the needs (and ability to pay) of the organizations.

\slide{Sample: Small Company}

\bd
\item[1-20 employees] 1-2 SAs; no formal helpdesk or tracking
\item[100-200] Formal helpdesk; determine future model for SA support
\ed

\slide{Sample: Medium Company}

\bi
\item Some specialists
\item Need dedicated helpdesk staff before 1000 employee-mark
\item Boundaries between the four categories may be fluid, and
  staff should be rotated
\ei

\slide{Sample: Large Company}

\bi
\item High degree of specialization
\item 3-tier (or more) model of SAs
\item Multiple architect roles
\item Central and regional SA support
\item Possibly even regional architects
\ei

\slide{Sample: E-commerce Site}

\bi
\item Separate internal from external support
\item Possibly consolidate with Help Desk
\ei

\slide{Sample: Non-Profits}

\bi
\item Very limited budgets for capital and run-rate
\item Possibly have access to low-cost labor (e.g., students or
  volunteers)
\item Often very political, with each group getting separate funding
\item Centralize as much as possible
\ei

\slide{Conclusion}

\bi
\item Focus on size and cost of the team versus its service
\item Understand management and funding chains 
\item Centralized versus decentralized
\item Consultants, contractors, and outsourcing
\ei

\end{document}
